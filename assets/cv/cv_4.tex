%%%%%%%%%%%%%%%%%%%%%%%%%%%%%%%%%%%%%%%%%
% Medium Length Professional CV
% LaTeX Template
% Version 2.0 (8/5/13)
% THIS SHOULD BE SAVED ZER
% This template has been downloaded from:
% http://www.LaTeXTemplates.com
%
% Original author:
% Trey Hunner (http://www.treyhunner.com/)
%
% Important note:
% This template requires the resume.cls file to be in the same directory as the
% .tex file. The resume.cls file provides the resume style used for structuring the
% document.
%
%%%%%%%%%%%%%%%%%%%%%%%%%%%%%%%%%%%%%%%%%

%----------------------------------------------------------------------------------------
%	PACKAGES AND OTHER DOCUMENT CONFIGURATIONS
%----------------------------------------------------------------------------------------

\documentclass{resume} % Use the custom resume.cls style

\usepackage[dvipsnames]{xcolor}
\usepackage[left=0.4 in,top=0.4in,right=0.4 in,bottom=0.4in]{geometry}
\usepackage{comment}
\usepackage{hyperref}
\usepackage[normalem]{ulem}
\hypersetup{
     colorlinks=true,
     linkcolor=blue,
     filecolor=blue,
     citecolor = black,      
     urlcolor=cyan,
     }
% Document margins
\newcommand{\tab}[1]{\hspace{.2667\textwidth}\rlap{#1}} 
\newcommand{\itab}[1]{\hspace{0em}\rlap{#1}}
\name{Nathanael Tepakbong} % Your name
\address{Hong Kong SAR, China $\vert$ + 852 64 14 04 21} % Your secondary address (optional)
\address{\href{mailto:nathanael.tepakbong-tematio@student.isae-supaero.fr}{\color{black}{nathanael.tepakbong-tematio@student.isae-supaero.fr}} $\vert$ \href{https://www.linkedin.com/in/ntepakbong}{\color{black}{linkedin.com/in/ntepakbong}}}

\begin{document}

\begin{comment}
%----------------------------------------------------------------------------------------
%	OBJECTIVE
%----------------------------------------------------------------------------------------

\begin{rSection}{Summary}

{- Final year Aerospace Engineering student specializing in Data Science, Statistics \& Probability Theory\\
- Currently following a Master's Degree in Mathematical Research on the study of Random Processes\\
- Eager to learn and use my skills to solve real-life problems in a research environment}
\end{rSection}
\end{comment}
%----------------------------------------------------------------------------------------
%	EDUCATION SECTION
%----------------------------------------------------------------------------------------

\begin{rSection}{EDUCATION}

{\textbf{Master's Degree in Mathematical Research and Innovation}}  \hfill 2020 - 2021\\
\textbf{\textcolor{blue}{Université Paul Sabatier}, Toulouse}\\
- Both degrees attended simultaneously. In-depth study of Probability Theory and Mathematical Statistics\\
- Relevant Coursework : Stochastic Calculus, Asymptotic Statistics, Statistical Learning

{\bf Master of Science in Aerospace Engineering (``Diplôme d'Ingénieur")} \hfill {2017 - 2021}
\\ 
\textbf{\textcolor{blue}{ISAE-Supaéro}, Toulouse}\\
- Leading ``Grande École" in Aerospace. Specialization in Applied Mathematics and Data Science \\
- Relevant Coursework : Advanced Statistics, Multi-Disciplinary Optimization, Algorithms in Machine Learning
 
{\textbf{Preparatory Classes in Mathematics, Physics and Computer Science}}  \hfill {2015 - 2017}\\
\textbf{\textcolor{blue}{Lycée Buffon}, Paris}\\
- Two-year intensive formation to prepare for the competitive entrance exams for the prestigious ``Grandes Écoles"\\
- Ended up being ranked in the top 5\% nationwide

{\textbf{High School Diploma with First Class Honours}}  \hfill {2015}\\
\textbf{\textcolor{blue}{Lycée Privé Saint-François de Sales}, Évreux}\\


%Minor in Linguistics \smallskip \\
%Member of Eta Kappa Nu \\
%Member of Upsilon Pi Epsilon \\


\end{rSection}

%----------------------------------------------------------------------------------------
%	WORK EXPERIENCE SECTION
%----------------------------------------------------------------------------------------



\begin{rSection}{EXPERIENCE}

%\begin{rSubsection}{Prof en Poche, Toulouse}{Oct 2020 - Apr 2021}{\textbf{\textcolor{blue}{Data Scientist (Part-time)}} }{}
%\item Training and Deployment of \textbf{Text-to-Speech} models for an educational app aimed at primary schoolers.
%\item Use of \textbf{Transfer Learning} algorithms to make a trained model able to display emotions.
%\item Models made with \textbf{PyTorch}. Training and Deployment done using \textbf{Docker} \& \textbf{Google Cloud Platform}.
%\end{rSubsection}
\begin{rSubsection}{City University of Hong Kong (Department of Mathematics), Hong Kong}{Jun 2021 - Dec 2021}{\textbf{\textcolor{blue}{Research Assistant}} }{}
\item Doing research on \textbf{Machine Learning Theory} through the framework of \textbf{Statistical Learning}.
\item Currently studying the Margin of \textbf{Feedforward Neural Networks} for \textbf{Functional Data Classification}.
\end{rSubsection}

\begin{rSubsection}{INRIA (Team SERENA), Paris}{Mar 2020 - Jul 2020}{\textbf{\textcolor{blue}{Research Intern}} }{}
\item Academic research laboratory specialized in \textbf{Computer Science} and \textbf{Applied Mathematics}
\item Developed and implemented \textbf{C++} parallel algorithms for efficient generation of \textbf{Gaussian Random Fields}
\item Findings written in a \textbf{submitted paper} with G. Pichot and S. Legrand : \href{https://hal.inria.fr/hal-03190252}{\color{black}{\uline{``Algorithms to speed up the generation of stationary Gaussian Random Fields with the Circulant Embedding method"} (2021)}}
\end{rSubsection}

\begin{rSubsection}{Robert Bosch Research and Technology Center, Singapore}{Mar 2019 - Sep 2019}{\textbf{\textcolor{blue}{R\&D Intern}} }{}
\item Corporate research laboratory developing \textbf{AI-based} solutions to air quality related problems
\item Designed from scratch using \textbf{C} and \textbf{Python} a real-time monitoring device for the level of pollution in a car cabin air filter, based on sensor readings and \textbf{Machine Learning} algorithms.
\end{rSubsection}

%\begin{rSubsection}{Space Advanced Concepts Laboratory (SACLAB), Toulouse}{Oct 2018 - Jan 2019}{\textbf{\textcolor{blue}{Research Assistant}} }{}
%\item Academic research laboratory studying numerous spatial navigation related problems.
%\item Developed new \textbf{MATLAB} and \textbf{Python} algorithms to visualize and optimize various kinds of spatial missions.
%\end{rSubsection}

\end{rSection} 



%----------------------------------------------------------------------------------------
%	TECHNICAL STRENGTHS SECTION
%----------------------------------------------------------------------------------------

\begin{rSection}{TECHNICAL SKILLS \& LANGUAGES}

\itab{Python (PyTorch, scikit-learn, pandas, matplotlib)} \tab{}  \tab{French - Native}
\\ \itab{MATLAB, R, Java, C, C++, SQL, \LaTeX} \tab{}  \tab{English - Fluent} 
\\ \itab{Parallel Computing : CUDA, MPI, OpenMP} \tab{}  \tab{German - Advanced (8 years)} 
\\ \itab{Git, Bash \& Shell Scripting, Docker} \tab{} \tab{Chinese (Mandarin) - HSK4 (4 years)}
\end{rSection}


%----------------------------------------------------------------------------------------

\begin{rSection}{HOBBIES}
\itab{Tutoring of High School \& Undergraduate students} \tab{}  \tab{Piano, Guitar \& Drums}
\\ \itab{Cooking \& Pastry (ex-head of culinary club)} \tab{}  \tab{Boxing \& Dancing}
\end{rSection}

\end{document}

